\documentclass[a4paper, 11pt]{article}

\usepackage{amsfonts}
\usepackage{amsmath}
\usepackage{amsthm}
\usepackage{appendix}
\usepackage{bm}
\usepackage{booktabs}
\usepackage[usenames, dvipsnames]{color}
\usepackage{graphicx}
\usepackage{epstopdf}
\epstopdfsetup{update}
\usepackage{helvet}
\usepackage{hyperref}
\usepackage{indentfirst}
\usepackage{lscape}
\usepackage{morefloats}
\usepackage{natbib} \bibliographystyle{ecta}
%\bibliographystyle{abbrvnat}\bibpunct{(}{)}{;}{a}{,}{,}
\usepackage{setspace}
\usepackage{subcaption}
\usepackage[capposition=top]{floatrow}
\usepackage{subfloat}
\usepackage[latin1]{inputenc}
\usepackage{tikz}
%\usepackage[pdf]{pstricks}

\usetikzlibrary{trees}
\usetikzlibrary{decorations.markings}


\theoremstyle{plain}
\newtheorem{thm}{Theorem}
\newtheorem{cor}{Corollary}
\newtheorem{lem}[thm]{Lemma}
\newtheorem{proposition}{Proposition}
\newtheorem{assumption}{Assumption}
\newtheorem{definition}{Definition}

%MARGINS
\topmargin   =  0.0in
\headheight  =  -0.3in
\headsep     =  0.7in
\oddsidemargin= 0.0in
\evensidemargin=0.0in
\textheight  =  9.0in
\textwidth   =  6.45in
% \setlength{\parindent}{4em}
\setlength{\parskip}{1em}

\newcommand{\fmt}{.eps}
%\newcommand{\fmt}{.png}

\hypersetup{
  colorlinks=true,
  linkcolor=BlueViolet,
  citecolor=BlueViolet,
  filecolor=BlueViolet,
  urlcolor=BlueViolet
}

\title{Gender and Quantity--Quality: Results from the Millennium Cohort Survey}
\author{Sonia Bhalotra \and Damian Clarke \and Patrick Donnelly Moran}
\date{\today}

\begin{document}
\sffamily
\maketitle

General point: With the MCS data we can look at the effect of additional births
on child outcomes in a much richer way.  We have measures of both
\textbf{Parental Investment} as well as much richer measures of
\textbf{child outcomes} which come from cognitive tests.  Also, the child
outcomes link quite well to the parental investment behaviours, so we can
test whether:

\[
\text{Additional Births}\rightarrow\text{Lower Parental Investments}\rightarrow\text{Worse Child Outcomes}
\] 

\noindent This is in line with Becker's original Q--Q formulation, in which the
trade-off is explicitly mediated by parental investment behaviour.  It also
has links to all of the newish papers on parental time use and child outcomes.

We are particularly interested in the gender dynamic here, as empirically it
seems like girls do worse when parents change their investment patterns after
birth (see \citet{Juhnetal2015}, as well as a large proportion of the results
from our Q--Q tests in other contexts).  We will thus estimate the following 
two stage least squares specification:
\begin{eqnarray}
  fertility_j = \alpha_1 + \alpha_2 twins_j + \bm{X} + \bm{S} + \bm{H} + \varepsilon_j \\
  y_{ij} = \beta_1 + \beta_2 \widehat{fertility}_j + \bm{X} + \bm{S} + \bm{H} + \varepsilon_j 
\end{eqnarray}
for child $i$ in family $j$.  This will be estimated seperately for male and female children.
Outcome variable $y$ will consist of the parental investment and child outcome variables in
the table below.
\begin{table}[htpb!]
\caption{Variables of Interest}
  \begin{tabular}{l} \toprule
  \textbf{Investment Variables}  \\
  Does parent read to child $i$  \\
  Does parent write with child $i$  \\
  \textbf{Outcome Variables}  \\
  Child $i$'s outcome on verbal tests \\
  Child $i$'s outcome on numerical tests \\
  Child $i$'s outcome on reading tests \\
  Child $i$'s outcome on pattern recognition tests \\
  \bottomrule
  \end{tabular}
\end{table}

Current things to follow up on:
\begin{enumerate}
\item What other outcomes do we have?  We have active school selection, though this has some issues, for example it may be that it is only chosen once.  In the UCL seminar, someone raised the interesting question of what happens to household goods?  Do parents spend more time on \emph{common} goods, and less time with each individual child?  Can we look at total time spent reading to all children, as well as total time spent reading to each child?  Can we estimate some elasticity of time use?
\item How do time dynamics work in this case?  We may expect a child who has had a twin sibling for longer should be more affected by the dilution of parental time
\item How do gender dynamics work?  Is a girl followed by girl twins less affected than a girl followed by boy twins?  All of this requires having sufficient power to split the sample in various ways
\item How does SES interact with this?  There is a mail from Sonia (Tues Sep 29, 2015 with topic ``MCS paper'') that we should look into
\end{enumerate}

\clearpage

\section{Tables}
\begin{table}[htbp]\centering
\def\sym#1{\ifmmode^{#1}\else\(^{#1}\)\fi}
\caption{Parental Investments and Q-Q Trade-off (two plus)}
\begin{tabular}{l*{4}{c}}
\toprule
                    &\multicolumn{2}{c}{Girls}      &\multicolumn{2}{c}{Boys}       \\\cmidrule(lr){2-3}\cmidrule(lr){4-5}
                    &Reading Help   &Writing Help   &Reading Help   &Writing Help   \\
\midrule
Fertility           &      -0.179   &      -0.599** &       0.189   &       0.232   \\
                    &     [0.281]   &     [0.293]   &     [0.307]   &     [0.349]   \\
\midrule
Observations        &        1704   &        1704   &        1744   &        1744   \\
\bottomrule\multicolumn{5}{p{12.6cm}}{\begin{footnotesize}
Notes: Reading Help measures the frequency with which a   
parent helps their child read during a five day week.     
Writing Help is measured similarly. Both are recorded in  
Wave 4 of the MCS. Standard errors are reported in parentheses. ***p-value$<$0.01, **p-value$<$0.05, *p-value$<$0.01.                                
\end{footnotesize}}\end{tabular}\end{table}


\begin{landscape}
\begin{table}[htbp]\centering
\def\sym#1{\ifmmode^{#1}\else\(^{#1}\)\fi}
\caption{Standardised Test Outcomes and Q-Q Trade-off (two plus)}
\begin{tabular}{l*{8}{c}}
\toprule
                    &\multicolumn{4}{c}{Girls}                                      &\multicolumn{4}{c}{Boys}                                       \\\cmidrule(lr){2-5}\cmidrule(lr){6-9}
                    &      Verbal   &       Maths   &     Reading   &    Patterns   &      Verbal   &       Maths   &     Reading   &    Patterns   \\
\midrule
Fertility           &      -0.585*  &      -0.560   &      -0.030   &      -1.236*  &       0.120   &       0.200   &       0.281   &      -0.041   \\
                    &     [0.307]   &     [0.362]   &     [0.314]   &     [0.631]   &     [0.264]   &     [0.296]   &     [0.403]   &     [0.265]   \\
\midrule
Observations        &        1796   &        1686   &        1664   &        1680   &        1868   &        1708   &        1688   &        1704   \\
\bottomrule\multicolumn{9}{p{15.8cm}}{\begin{footnotesize}
Notes: Verbal score is from the British Ability Scales,   
Second Edition, measured in Wave 5 of the MCS.            
Mathematical ability comes from NFER Number Skills,       
measured in Wave 4 of the MCS. Word Reading and Pattern   
Construction both come from the British Ability Scales in 
Wave 4 of the MCS. Standard errors are reported in parentheses. ***p-value$<$0.01, **p-value$<$0.05, *p-value$<$0.01.                                
\end{footnotesize}}\end{tabular}\end{table}

\end{landscape}

\begin{table}[htbp]\centering
\def\sym#1{\ifmmode^{#1}\else\(^{#1}\)\fi}
\caption{Parental Investments and Q-Q Trade-off}
\begin{tabular}{l*{4}{c}}
\toprule
                    &\multicolumn{2}{c}{Girls}      &\multicolumn{2}{c}{Boys}       \\\cmidrule(lr){2-3}\cmidrule(lr){4-5}
                    &Reading Help   &Writing Help   &Reading Help   &Writing Help   \\
\midrule
Fertility           &      -0.540   &      -0.722   &       0.088   &       0.094   \\
                    &     [0.673]   &     [0.610]   &     [0.230]   &     [0.271]   \\
\midrule
Observations        &        1284   &        1284   &        1324   &        1324   \\
\bottomrule\multicolumn{5}{p{14.6cm}}{\begin{footnotesize}        
Notes: Reading Help measures the frequency with which a parent helps their child read during a five day week. Writing Help is measured similarly. Both are recorded in Wave 4 of the MCS. Standard errors are reported in parentheses. ***p-value$<$0.01, **p-value$<$0.05, *p-value$<$0.01.                                
\end{footnotesize}}\end{tabular}\end{table}


\begin{landscape}
\begin{table}[htbp]\centering
\def\sym#1{\ifmmode^{#1}\else\(^{#1}\)\fi}
\caption{Standardised Test Outcomes and Q-Q Trade-off (three plus)}
\begin{tabular}{l*{8}{c}}
\toprule
                    &\multicolumn{4}{c}{Girls}                                      &\multicolumn{4}{c}{Boys}                                       \\\cmidrule(lr){2-5}\cmidrule(lr){6-9}
                    &      Verbal   &       Maths   &     Reading   &    Patterns   &      Verbal   &       Maths   &     Reading   &    Patterns   \\
\midrule
Fertility           &      -0.850** &      -1.221** &      -0.122   &       0.544   &       0.096   &      -0.191   &      -0.211   &       0.154   \\
                    &     [0.398]   &     [0.483]   &     [0.496]   &     [0.732]   &     [0.217]   &     [0.228]   &     [0.210]   &     [0.174]   \\
\midrule
Observations        &        1365   &        1268   &        1251   &        1263   &        1428   &        1293   &        1273   &        1287   \\
\bottomrule\multicolumn{9}{p{15.8cm}}{\begin{footnotesize}
Notes: Verbal score is from the British Ability Scales,   
Second Edition, measured in Wave 5 of the MCS.            
Mathematical ability comes from NFER Number Skills,       
measured in Wave 4 of the MCS. Word Reading and Pattern   
Construction both come from the British Ability Scales in 
Wave 4 of the MCS. Standard errors are reported in parentheses. ***p-value$<$0.01, **p-value$<$0.05, *p-value$<$0.01.                                
\end{footnotesize}}\end{tabular}\end{table}
\end{landscape}

\clearpage
\bibliography{./refs}


\end{document}
